\chapter{Calcolo combinatorio}
\section{Permutazioni}
\begin{defn}[Fattoriale]
	Dato un numero naturale, diremo fattoriale di un numero e indicato con $n!$ il numero:
	\begin{align*}
	n!=&n\cdot(n-1)\cdot (n-2),\dots,2\cdot 1\\
	0!=&1\\
	1!=&1
	\end{align*}
\end{defn}\index{Fattoriale}
Quando disponiamo degli oggetti in gruppi è importante distinguere in vari casi. Importa l'ordine con cui operiamo? Vi sono ripetizioni? Tutto questo dipende dalle situazioni. Se giochiamo a tombola no ma se pensiamo a come sono fatte le targhe di un'auto si. Iniziamo da problemi di ordine.
\begin{defn}[Permutazioni semplici]
Si dicono permutazioni di $n$ oggetti distinti tutti gli insiemi di $n$ oggetti distinti. Consideriamo distinti due insiemi se differiscono per l'ordine con cui sono formati. Indicheremo con $P_n$ il numero di questi insiemi.\index{Permutazioni!semplici}
\end{defn}
\begin{table}
	\centering
	\begin{tabular}{ccc}
		$N$	& $P$ &$P_n$  \\
		\toprule
		1	&a  &1  \\
		\midrule
		2	&\begin{tabular}{cc}
			ab	& ba   \\
		\end{tabular} &2  \\
		\midrule
		3	&\begin{tabular}{cc}
			abc	& bac   \\
			acb	& bca   \\
			cab	& aba   \\	
		\end{tabular} &6  \\
		\midrule
		4	&\begin{tabular}{ccc}
			abcd	& cabd &bacd  \\
			abdc	& cadb &badc  \\		
			adbc	& cdab &bdac  \\
			dabc	& daab &daac  \\
			acbd	& bacd &cbad  \\
			acdb	& badc &cbda  \\
			adcb	& bdac &cdba  \\
			dacb	& dbac &dcba  \\
		\end{tabular} &24  \\
	\bottomrule
	\end{tabular}
	\caption{Permutazioni}
	\label{tab:permutazioni}
\end{table}
\begin{figure}
	\centering
	\documentclass[10pt,a4paper]{standalone}
\usepackage[utf8]{inputenc}
\usepackage[T1]{fontenc}
\usepackage[italian]{babel}
\usepackage{tikz}
\usetikzlibrary{arrows}
\begin{document}
	\begin{tikzpicture}[nodes={draw,  circle}, ->=triangle 45, thick,grow=right]
	
\node{}
child { node {3} 
child { node {5} 
child { node {9} }
}
child { node {9} 
child { node {5} }
}
}
 child [missing] 
child { node {5} 
child { node {9}
child { node {3}}
}
child { node {3}
child { node {9}}
}
}
 child [missing]
child { node {9} 
child { node {5}
child { node {3}}
}
child { node {3}
child { node {5}}
}
};

	
	\end{tikzpicture}
\end{document}
	\caption{Albero permutazioni}
	\label{fig:alberopermutatre}
\end{figure}
\begin{exmp}
Dati tre numeri \numlist{3;5;9}, quali e quante sono le loro permutazioni? 
\end{exmp}
Consideriamo la~\vref{fig:alberopermutatre}. Come prima scelta possiamo utilizzare tre valori, al secondo livello possiamo scegliere tra due valori. Al terzo livello possiamo scegliere un solo valore. Quindi le permutazioni sono \numlist{359;395;593;539;953;935}.
\begin{exmp}
	Quali e quante sono le permutazioni di quattro elementi?
\end{exmp}
Come dimostra la~\vref{tab:permutazioni} con un elemento abbiamo una permutazione. Con due due. Con tre diventano sei e infine con quattro, abbiamo 24 permutazioni.
\begin{thm}[Numero delle permutazioni]
	Il numero delle permutazioni di $n$ oggetti è uguale a
\[P_n=n\cdot (n-1)\cdot(n-2)\dots 2\cdot 1\]
\end{thm}
\begin{proof}
Se abbiamo $n$ oggetti per calcolare $P_n$ possiamo procedere in questo modo. Supponiamo di avere $n$ scatole vuote tante quante sono gli oggetti.\par 
\makebox[\linewidth]{\scatola{}\scatola{}\scatola{}\scatola{},\dots,\scatola{}}\par
 All'inizio abbiamo $n$ oggetti quindi possiamo scegliere fra $n$ per inserirne uno nella prima\par 
\makebox[\linewidth]{\scatola{n}\scatola{}\scatola{}\scatola{},\dots,\scatola{}}\par
 Dopo questa scelta rimangono $n-1$ oggetti tra cui scegliere quindi avremo:\par 
\makebox[\linewidth]{\scatola{n}\scatola{n-1}\scatola{}\scatola{},\dots,\scatola{}}\par
 Per la terza scatola restano $n-2$ oggetti. Quindi\par 
\makebox[\linewidth]{\scatola{n}\scatola{n-1}\scatola{n-2}\scatola{},\dots,\scatola{}}\par
 Per la quarta $n-3$ etc.\par \makebox[\linewidth]{\scatola{n}\scatola{n-1}\scatola{n-2}\scatola{n-3},\dots,\scatola{}}\par Per l'ultima scatola resta una sola scelta. \par \makebox[\linewidth]{\scatola{n}\scatola{n-1}\scatola{n-2}\scatola{n-3},\dots,\scatola{1}}\par Quindi \[P_n=n\cdot (n-1)\cdot(n-2)\dots 2\cdot 1\]
\end{proof}
\section{Disposizioni semplici}
\begin{defn}[Disposizioni semplici]
Diremo disposizioni semplici di $n$ oggetti distinti a gruppi di $k$ ($ k\leqslant n$), tutti i raggruppamenti di $k$ oggetti tali che ogni gruppo differisce o per l'ordine o per almeno un oggetto. Indicheremo con $D_{n,k}$ il numero di queste disposizioni.
\end{defn}\index{Disposizioni!semplici}
\begin{figure}
	\centering
	\includestandalone{grafici/DispSemplici}
	\caption{Disposizioni semplici}
	\label{fig:dispsemplici}
\end{figure}
