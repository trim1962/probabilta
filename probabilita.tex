% !TeX root = probabilita.tex
% !BIB TS-program = biber
% !TeX encoding = UTF-8
% !TeX spellcheck = it_IT

\documentclass[a4paper,oneside]{book}%
\usepackage{cmap}
\usepackage[big]{layaureo}
\usepackage{copyright}
\frenchspacing%
\usepackage{amsmath}

\usepackage{amssymb}
\usepackage[italian]{babel}
\usepackage[thmmarks,hyperref]{ntheorem}
\usepackage{miamatematica}

\usepackage{lmodern} % load vector font
\usepackage[T1]{fontenc} % font encoding
\usepackage[utf8]{inputenc} % input encoding
%\usepackage{noto}
\usepackage[babel=true]{microtype}
%\usepackage{geometry}
\usepackage{textcomp}

%\geometry{top=1.5cm,bottom=1.5cm}
\usepackage{grafica}

%Teorema
\theoremstyle{marginbreak}
\theoremheaderfont{\normalfont\bfseries}\theorembodyfont{\slshape}
\theoremsymbol{\ensuremath{\diamondsuit}}
\theoremseparator{:} %
\newtheorem{thm}{Teorema}[section]
%Proprietà
\theoremstyle{marginbreak}
\theoremheaderfont{\normalfont\bfseries}\theorembodyfont{\slshape}
\theoremsymbol{\ensuremath{\diamondsuit}}
\theoremseparator{:}
\newtheorem{prop}{Proprietà}[section]
%lemma
\theoremstyle{changebreak}
\theoremsymbol{\ensuremath{\heartsuit}}
\theoremindent0.5cm
\theoremnumbering{greek}
\newtheorem{lem}[thm]{Lemma}
%corollario
\theoremindent0cm
\theoremsymbol{\ensuremath{\spadesuit}}
\theoremnumbering{arabic}
\newtheorem{cor}[thm]{Corollario}
%esempio
\theoremstyle{change}
\theorembodyfont{\upshape}
\theoremsymbol{\ensuremath{\ast}}
\theoremseparator{}
\newtheorem{exmp}{Esempio}[section]
%controesempio
\theoremstyle{change}
\theorembodyfont{\upshape}
\theoremsymbol{\ensuremath{\odot}}
\theoremseparator{}
\newtheorem{cexmp}{Contro esempio}[section]
%definizione
\theoremstyle{plain}
\theoremsymbol{\ensuremath{\clubsuit}}
\theoremseparator{.}
\theoremstyle{marginbreak}
\theoremprework{\hrule\bigskip}
\theorempostwork{\hrule\bigskip}
\newtheorem{defn}{Definizione}[section]
%commento
\theoremstyle{plain}
\theorembodyfont{\upshape}
\theoremsymbol{\ensuremath{\blacklozenge}}
\theoremseparator{:}
\newtheorem{commento}{Commento}
%dimostrazione

\theoremstyle{plain}
\theoremheaderfont{\sc}
\theorembodyfont{\bfseries}
\theoremstyle{nonumberplain}
%^{}\theoremseparator{.}

\theoremsymbol{\ensuremath{\blacksquare}}
\theoremheaderfont{\bfseries}
%\theoremstyle{nonumberplain}
%\theoremstyle{marginbreak}
\theorembodyfont{\normalfont}
\newtheorem{proof}{Dimostrazione}
%\input{../Mod_base/tabelle}



%\usepackage{adjustbox}
%\input{../Mod_base/stand_class}
\usepackage{pagina}

\setlength{\headheight}{13pt}
\usepackage{indice}
\usepackage{date}
\usepackage{unita_misura}

\usepackage{imakeidx}
\makeindex[options=-s ../Mod_base/oldclaudio.sti]

\usepackage{diagbox}
%\include{simboli_operatori}

\usepackage{stand_class}

\newcommand{\HRule}{\rule{\linewidth}{0.5mm}}


 \makeatletter
 \renewcommand\frontmatter{%
 	\cleardoublepage
 	\@mainmatterfalse
 	%\pagenumbering{roman}
 }
 \renewcommand\mainmatter{%
 	\cleardoublepage
 	\@mainmattertrue
 	%\pagenumbering{arabic}
 }
 \makeatother

\usepackage[grumpy,mark,markifdirty,raisemark=0.95\paperheight]{gitinfo2}
% 10/02/2018 :: 20:05:58 :: \usepackage{parskip}
\usepackage[toc,page]{appendix}

\renewcommand{\appendixtocname}{Appendici}

\renewcommand{\appendixpagename}{Appendici}


%\usepackage[style=italian]{csquotes}
%\usepackage[%
%style=philosophy-modern,
%annotation=true,
%hyperref,
%backend=biber,
%backref]{biblatex}
%\addbibresource{formulario.bib}
\usepackage[italian]{varioref}
\usepackage{hyperxmp}
\usepackage[pdfpagelabels]{hyperref}
\usepackage[italian,noabbrev]{cleveref}
\crefname{defn}{definizione}{definizioni}
\Crefname{defn}{Definizione}{Definizioni}
\crefname{thm}{teorema}{teoremi}
\Crefname{thm}{Teorema}{Teoremi}
\crefname{cor}{corollario}{corollari}
\Crefname{cor}{Corollario}{Corollari}
\crefname{equation}{equazione}{equazioni}
\Crefname{equation}{Equazione}{Equazioni}
\crefname{sistema}{sistema}{sistemi}
\Crefname{sistema}{Sistema}{Sistemi}
\crefname{lem}{lemma}{lemmi}
\Crefname{lem}{Lemma}{Lemmi}
\creflabelformat{equation}{#2\textup{#1}#3}

%\usepackage{tcolorboxgest}
\title{Probabilità}
\author{Claudio Duchi}
\date{\datetime}
\hypersetup{%
pdfencoding=auto,
urlcolor={blue},
pdftitle={Probalità},
pdfsubject={Per non dimenticare},
pdfstartview={FitH},
pdfpagemode={UseOutlines},
pdflicenseurl={http://creativecommons.org/licenses/by-nc-nd/3.0/},
pdflang={it},
pdfmetalang={it},
pdfkeywords={Calcolo combinatorio, probabilità},
pdfcopyright={Copyright (C) 2020, Claudio Duchi},
pdfcontacturl={http://breviariomatematico.altervista.org},
pdfcontactpostcode={06128},
pdfcontactphone={},
pdfcontactemail={claduc},
pdfcontactcountry={Italy},
pdfcontactcity={Perugia},
pdfcontactaddress={},
pdfcaptionwriter={Claudio Duchi},
pdfauthortitle={},%
pdfauthor={Claudio Duchi},
linkcolor={blue},
colorlinks=true,
citecolor={red},
breaklinks,
bookmarksopen,
verbose,
baseurl={http://breviariomatematico.altervista.org}
}

% !TeX root = Asparsi.tex
% !BIB TS-program = biber
% !TeX encoding = UTF-8
% !TeX spellcheck = it_IT
\includeonly{%
calcolocomb
}


%patch allieamento lista teoremi
\usepackage{regexpatch}
\makeatletter
%\xpatchcmd*{\thm@@thmline}{2.3em}{5em}{}{} % not really needed
\xpatchcmd*{\thm@@thmline@name}{2.3em}{5em}{}{} 
\xpatchcmd*{\thm@@thmline@noname}{2.3em}{5em}{}{}
\makeatother
%fine patch allieamento lista teoremi
\usepackage{CDloghi}
\listfiles
\begin{document}
\begin{titlepage}
\begin{center}	
	\Lgrandedue\\[1cm]    
	\textsc{\LARGE Claudio Duchi}\\[1.4cm]
	\HRule \\[0.4cm]
{ \huge \bfseries Probabilità}\\[0.4cm]
%{ \large \bfseries di}\\[0.4cm]
%{ \huge \bfseries Pensieri}\\[0.4cm]
\HRule \\
\vfill
	% Bottom of the page
	\polylogo[5.5]{9}		
		{\large $-$\DTMnow$-$}	
\end{center}
{\centering
Release:\gitReln\ (\gitAbbrevHash)\ Autore:\gitAuthorName\ 
\gitCommitterDate \\
}
\end{titlepage}	
	\hypersetup{pageanchor=true}
		\CDcopyright
		\tableofcontents
		\chapter*{Lista dei teoremi}
		\theoremlisttype{allname}
		\listtheorems{thm,defn,cor,comm,lem}
	\addcontentsline{toc}{chapter}{\listfigurename}%
		\listoffigures
%	\addcontentsline{toc}{chapter}{\listtablename}%
%			\listoftables
			\mainmatter

% !TeX root = Asparsi.tex
% !BIB TS-program = biber
% !TeX encoding = UTF-8
% !TeX spellcheck = it_IT
\chapter{Calcolo combinatorio}
\section{Permutazioni}
\begin{defn}[Fattoriale]
	Dato un numero naturale, diremo fattoriale di un numero e indicato con $n!$ il numero:
	\begin{align*}
	n!=&n\cdot(n-1)\cdot (n-2),\dots,2\cdot 1\\
	0!=&1\\
	1!=&1
	\end{align*}
\end{defn}\index{Fattoriale}
Quando disponiamo degli oggetti in gruppi è importante distinguere in vari casi. Importa l'ordine con cui operiamo? Vi sono ripetizioni? Tutto questo dipende dalle situazioni. Se giochiamo a tombola no ma se pensiamo a come sono fatte le targhe di un'auto si. Iniziamo da problemi di ordine.
\begin{defn}[Permutazioni semplici]
Si dicono permutazioni di $n$ oggetti distinti tutti gli insiemi di $n$ oggetti distinti. Consideriamo distinti due insiemi se differiscono per l'ordine con cui sono formati. Indicheremo con $P_n$ il numero di questi insiemi.\index{Permutazioni!semplici}
\end{defn}
\begin{table}
	\centering
	\begin{tabular}{ccc}
		$N$	& $P$ &$P_n$  \\
		\toprule
		1	&a  &1  \\
		\midrule
		2	&\begin{tabular}{cc}
			ab	& ba   \\
		\end{tabular} &2  \\
		\midrule
		3	&\begin{tabular}{cc}
			abc	& bac   \\
			acb	& bca   \\
			cab	& aba   \\	
		\end{tabular} &6  \\
		\midrule
		4	&\begin{tabular}{ccc}
			abcd	& cabd &bacd  \\
			abdc	& cadb &badc  \\		
			adbc	& cdab &bdac  \\
			dabc	& daab &daac  \\
			acbd	& bacd &cbad  \\
			acdb	& badc &cbda  \\
			adcb	& bdac &cdba  \\
			dacb	& dbac &dcba  \\
		\end{tabular} &24  \\
	\bottomrule
	\end{tabular}
	\caption{Permutazioni}
	\label{tab:permutazioni}
\end{table}
\begin{figure}
	\centering
	\documentclass[10pt,a4paper]{standalone}
\usepackage[utf8]{inputenc}
\usepackage[T1]{fontenc}
\usepackage[italian]{babel}
\usepackage{tikz}
\usetikzlibrary{arrows}
\begin{document}
	\begin{tikzpicture}[nodes={draw,  circle}, ->=triangle 45, thick,grow=right]
	
\node{}
child { node {3} 
child { node {5} 
child { node {9} }
}
child { node {9} 
child { node {5} }
}
}
 child [missing] 
child { node {5} 
child { node {9}
child { node {3}}
}
child { node {3}
child { node {9}}
}
}
 child [missing]
child { node {9} 
child { node {5}
child { node {3}}
}
child { node {3}
child { node {5}}
}
};

	
	\end{tikzpicture}
\end{document}
	\caption{Albero permutazioni}
	\label{fig:alberopermutatre}
\end{figure}
\begin{exmp}
Dati tre numeri \numlist{3;5;9}, quali e quante sono le loro permutazioni? 
\end{exmp}
Consideriamo la~\vref{fig:alberopermutatre}. Come prima scelta possiamo utilizzare tre valori, al secondo livello possiamo scegliere tra due valori. Al terzo livello possiamo scegliere un solo valore. Quindi le permutazioni sono \numlist{359;395;593;539;953;935}.
\begin{exmp}
	Quali e quante sono le permutazioni di quattro elementi?
\end{exmp}
Come dimostra la~\vref{tab:permutazioni} con un elemento abbiamo una permutazione. Con due due. Con tre diventano sei e infine con quattro, abbiamo 24 permutazioni.
\begin{thm}[Numero delle permutazioni]
	Il numero delle permutazioni di $n$ oggetti è uguale a
\[P_n=n\cdot (n-1)\cdot(n-2)\dots 2\cdot 1\]
\end{thm}
\begin{proof}
Se abbiamo $n$ oggetti per calcolare $P_n$ possiamo procedere in questo modo. Supponiamo di avere $n$ scatole vuote tante quante sono gli oggetti.\par 
\makebox[\linewidth]{\scatola{}\scatola{}\scatola{}\scatola{},\dots,\scatola{}}\par
 All'inizio abbiamo $n$ oggetti quindi possiamo scegliere fra $n$ per inserirne uno nella prima\par 
\makebox[\linewidth]{\scatola{n}\scatola{}\scatola{}\scatola{},\dots,\scatola{}}\par
 Dopo questa scelta rimangono $n-1$ oggetti tra cui scegliere quindi avremo:\par 
\makebox[\linewidth]{\scatola{n}\scatola{n-1}\scatola{}\scatola{},\dots,\scatola{}}\par
 Per la terza scatola restano $n-2$ oggetti. Quindi\par 
\makebox[\linewidth]{\scatola{n}\scatola{n-1}\scatola{n-2}\scatola{},\dots,\scatola{}}\par
 Per la quarta $n-3$ etc.\par \makebox[\linewidth]{\scatola{n}\scatola{n-1}\scatola{n-2}\scatola{n-3},\dots,\scatola{}}\par Per l'ultima scatola resta una sola scelta. \par \makebox[\linewidth]{\scatola{n}\scatola{n-1}\scatola{n-2}\scatola{n-3},\dots,\scatola{1}}\par Quindi \[P_n=n\cdot (n-1)\cdot(n-2)\dots 2\cdot 1\]
\end{proof}
\section{Disposizioni semplici}
\begin{defn}[Disposizioni semplici]
Diremo disposizioni semplici di $n$ oggetti distinti a gruppi di $k$ ($ k\leqslant n$), tutti i raggruppamenti di $k$ oggetti tali che ogni gruppo differisce o per l'ordine o per almeno un oggetto. Indicheremo con $D_{n,k}$ il numero di queste disposizioni.
\end{defn}\index{Disposizioni!semplici}
\begin{figure}
	\centering
	\includestandalone{grafici/DispSemplici}
	\caption{Disposizioni semplici}
	\label{fig:dispsemplici}
\end{figure}
\begin{thm}[Numero delle disposizioni]
	Il numero delle disposizioni semplici di $n$ oggetti a gruppi di $k$ $(k\leqslant n)$ è uguale a
	\[D_{n,k}=n(n-2)(n-3)\dots(n-k+1)=\dfrac{n!}{(n-k)!}\]
\end{thm}\index{Disposizioni!semplici}
\begin{proof}
	Se abbiamo $n$ elementi, per calcolare le $D_{n,k}$  procediamo in questo modo. Supponiamo di avere $k$ scatole vuote.\par 
	\makebox[\linewidth]{\scatola{}\scatola{}\scatola{}\scatola{},\dots,\scatola{}}\par
	All'inizio abbiamo $n$ oggetti quindi possiamo sceglierne $n$ per inserirne uno nella prima scatola\par 
	\makebox[\linewidth]{\scatola{n}\scatola{}\scatola{}\scatola{},\dots,\scatola{}}\par
	Dopo questa scelta rimangono $n-1$ oggetti tra cui scegliere quindi avremo:\par 
	\makebox[\linewidth]{\scatola{n}\scatola{n-1}\scatola{}\scatola{},\dots,\scatola{}}\par
	Per la terza scatola restano $n-2$ oggetti. Quindi\par 
	\makebox[\linewidth]{\scatola{n}\scatola{n-1}\scatola{n-2}\scatola{},\dots,\scatola{}}\par
	Per la quarta $n-3$ etc.\par \makebox[\linewidth]{\scatola{n}\scatola{n-1}\scatola{n-2}\scatola{n-3},\dots,\scatola{}}\par Per l'ultima scatola restano $n-k+1$ oggetti  quindi \par \makebox[\linewidth]{\scatola{n}\scatola{n-1}\scatola{n-2}\scatola{n-3},\dots,\scatola{n-k+1}}\par Quindi \[D_{n,k}=n(n-2)(n-3)\dots(n-k+1)=\dfrac{n!}{(n-k)!}\]
	\begin{align*}
	D_{n,k}=&n(n-2)(n-3)\dots(n-k+1)\\
	=&\dfrac{n(n-2)(n-3)\dots(n-k+1)(n-k)\dots 2\cdot 1}{(n-k)!}\\
	=&\dfrac{n!}{(n-k)!}
	\end{align*}
	Come volevasi dimostrare
\end{proof}
\begin{exmp}
	Dati quattro oggetti distinti $\lbrace a,b,c,d \rbrace$ quante e quali sono le disposizioni semplici di quattro elementi a gruppi di due? 
\end{exmp}
Per risolvere basta considerare~\vref{fig:dispsemplici}. L'albero mostra tutte le possibili disposizioni di quattro elementi prima in gruppi di uno poi in gruppi di due.
\section{Disposizioni con ripetizione} 


			

\nocite{*}
% \addcontentsline{toc}{chapter}{\bibname}
%\printbibliography
 \addcontentsline{toc}{chapter}{\indexname}
 \printindex
 \appendix
 \chapter{Mezzi usati}
 \CDMezziUsati
\end{document}
